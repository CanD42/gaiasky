%%% DOCUMENTCLASS 
%%%-------------------------------------------------------------------------------

\documentclass[
a4paper, % Stock and paper size.
11pt, % Type size.
% article,
% oneside, 
onecolumn, % Only one column of text on a page.
% openright, % Each chapter will start on a recto page.
% openleft, % Each chapter will start on a verso page.
openany, % A chapter may start on either a recto or verso page.
]{memoir}

%%% PACKAGES 
%%%------------------------------------------------------------------------------

\usepackage[utf8]{inputenc} % If utf8 encoding
\usepackage[T1]{fontenc}    %
\usepackage[english]{babel} % English please
\usepackage[final]{microtype} % Less badboxes
\usepackage{amsmath,amssymb,mathtools} % Math
\usepackage{graphicx} % Include figures
\usepackage{hyperref}
\usepackage{listings}
\usepackage{color}
\usepackage{titlesec}
\usepackage{etoolbox}
\usepackage{url}

\definecolor{dkgreen}{rgb}{0,0.6,0}
\definecolor{gray}{rgb}{0.5,0.5,0.5}
\definecolor{mauve}{rgb}{0.58,0,0.82}

\lstset{frame=tb,
  aboveskip=5mm,
  belowskip=5mm,
  showstringspaces=false,
  columns=flexible,
  basicstyle={\small\ttfamily},
  numbers=none,
  numberstyle=\tiny\color{gray},
  keywordstyle=\color{blue},
  commentstyle=\color{dkgreen},
  stringstyle=\color{mauve},
  breaklines=true,
  breakatwhitespace=true
  tabsize=3
}

\hypersetup{
  colorlinks,
  citecolor=violet,
  linkcolor=blue,
  urlcolor=blue}

%%% PAGE LAYOUT 
%%%------------------------------------------------------------------------------

\setlrmarginsandblock{0.15\paperwidth}{*}{1} % Left and right margin
\setulmarginsandblock{0.2\paperwidth}{*}{1}  % Upper and lower margin
\checkandfixthelayout


%%% FONTS
%%%------------------------------------------------------------------------------

\fontencoding{T1}
\fontfamily{pandora}
\fontseries{m}
\fontshape{it}
\fontsize{11}{13}
\selectfont


%%% HEADING STYLES
%%%------------------------------------------------------------------------------

\makeatother
\titleformat{\chapter}
  {\normalfont\LARGE\bfseries\color{black}}{\thechapter.}{.8em}{}

%%% NO INDENT
%%%------------------------------------------------------------------------------

\newlength\tindent
\setlength{\tindent}{\parindent}
\setlength{\parindent}{0pt}
\renewcommand{\indent}{\hspace*{\tindent}}


%%% DOCUMENT
%%%------------------------------------------------------------------------------

\title{Gaia Sandbox user manual \\
\small
\url{http://www.zah.uni-heidelberg.de/gaia2/outreach/gaiasandbox}}
\author{
        Toni Sagrist\`a Sell\'es \\
        Astronomisches Rechen-Institut\\
	Zentrum f\"ur Astronomie Heidelberg \\
	UNIVERSIT\"AT HEIDELBERG \\
        Heidelberg, Baden-W\"urttemberg, Deutschland
}
\date{\today}


\begin{document}
\maketitle
\small

\tableofcontents
\newpage

\chapter{Introduction}
The Gaia Sandbox is a visualisation application developed in the Gaia group of Astronomisches Rechen-Institut 
with the purpose of serving as outreach software so that the general public can inspect various aspects of the
Gaia mission in an interactive manner. With this software one can discover where Gaia is in its orbit at any time,
inspect its movement and attitude patterns and behold the satellite model and its parts. Additionally, the HYG
catalog loaded into the application allows for the real time visualisation and exploration of more than 100.000 stars
in the vicinity of the Sun. The application is packed with features and details, some of which are covered in
this user manual document. 


\chapter{System requirements and installation}\label{sec:installationrequirements}

\section{System requirements}

This section describes the system requirements you have to meet in order to run the Gaia Sandbox.

\subsection{Operating systems}
The Gaia Sandbox application runs on Windows, MacOS and
Linux. So far it has only been tested to work 
on Ubuntu 13.4, Ubuntu 13.10, Linux Mint 16 and Windows 7 Ultimate.

\subsection{Java}
In order to run this software you will need the Java
Runtime Environment (JRE) 7+ installed in your system.
We recommend using the Oracle HotSpot JVM
(\href{http://www.oracle.com/technetwork/java/javase/downloads/index.html}{http://www.oracle.com/java}),
as it is the only one we test, but it may also run
in IBM's JVM or in the OpenJDK JVM.
It may also run in Java 6, but it has not been 
tested.

\subsection{OpenGL}
You will need an operating system that supports at least OpenGL 2.0.

\subsection{Hardware requirements}
We have not tested the application on many different
machines, but for sure you weed a decent GPU
and an average CPU to get good frame rates. Specific 
system requirements are yet to be determined.

\section{Installation}
The Gaia Sandbox application does not require a 'formal'
installation, as it is ready to be executed out of the
box. Just unzip it anywhere in your desired drive and
that's about it.

\chapter{Configuration options}\label{sec:config}
The configuration file is \texttt{conf/global.properties}. 
This file contains some useful configuration parameters 
to tweak the application:

\section{Resolution}
You may change the resolution of the application. By default it 
is configured to execute at 1280x720 (16:9 aspect ratio). To change
the running resolution you need to modify the following properties
in the configuration file.

\begin{lstlisting}[language=Java]
graphics.screen.width=1280
graphics.screen.height=720
\end{lstlisting}

\section{Mode}
By default the application will execute in windowed mode. You can
switch it to full screen by setting the following property to true:

\begin{lstlisting}[language=Java]
graphics.screen.fullscreen=false
\end{lstlisting}

However, not all resolutions can be used with the full-screen mode.
The list of native resolutions is outputted in the console when
the application is started, just under the "Available full screen
display modes" text. The output is something like this:

\begin{lstlisting}[language=Java]
1600x1200, bpp: -1, hz: 50
1280x1024, bpp: -1, hz: 51
1024x768, bpp: -1, hz: 53
800x600, bpp: -1, hz: 55
\end{lstlisting}

These are the resolutions you can use in full-screen. If you use any other
resolution the program will crash.

\section{Anti-aliasing}
MSAA (multi-sample anti-aliasing) is deactivated by default. However, if you
have a fast enough graphics card you can activate it by specifying the
number of samples to 2, 4, 8 or 16. The more samples the better the quality of 
MSAA.
\begin{lstlisting}[language=Java]
graphics.screen.antialiasing=0
\end{lstlisting}

\section{Rendering to images}
The application can also render to image files. In order to configure the 
rendering to images one must modify the properties that start with \texttt{graphics.render}.
One may modify the resolution, the target FPS, the output folder, the 
image name prefix and the anti-aliasing.

\begin{lstlisting}[language=Java]
graphics.render.renderoutput=false
graphics.render.width=1600
graphics.render.height=900
graphics.render.targetfps=24
graphics.render.folder=/home/tsagrista/tmp/OrbitVideo/5/
graphics.render.filename=OrbitVideo
graphics.render.antialiasing=false
\end{lstlisting}

\section{Orbit data configuration}
The orbit data file can be configured as well. Usually you'll never need 
to touch this as we will bundle newer data with subsequent versions of 
this application. One can also deactivate the heliotropic correction.

\begin{lstlisting}[language=Java]
data.orbit.file=data/ORB1_20131127_000001.topcat
data.orbit.referenceframe=RF_HELIOTROPIC
\end{lstlisting}

The data orbit file is an ASCII file with the following columns:\\

\texttt{DATE X Y Z X\_dot Y\_dot Z\_dot X\_dotdot Y\_dotdot Z\_dotdot}\\

Where the date is in the format \texttt{yyyy-MM-ddThh:mm:ss.ss}, the positions
\texttt{X}, \texttt{Y} and \texttt{Z} are in $km$, the velocities \texttt{X\_dot}, \texttt{Y\_dot}, \texttt{Z\_dot}
are in $km/s$ and the accelerations are in $km/s^2$.

\section{Catalog file}
The application is shipped with the HYG catalog (\href{http://www.astronexus.com/hyg}{http://www.astronexus.com/hyg})
in an own binary form. However, it is possible to use other catalogs by either
using the provided loaders (described in the config file) or by adding your
own. You just need to implement the ICatalogLoader interface with its
\texttt{loadStars()} method that returns a list of Star objects.
Then, you need to point the program to the file you want to load.

\begin{lstlisting}[language=Java]
data.catalog.loader=gaia.cu9.ari.gaiaorbit.catalog.HYGBinaryLoader
data.catalog.file=data/hygxyz.bin
\end{lstlisting}

\section{Scene properties}
You can change some properties of the scene itself such as the camera field
of view, the velocity in focus mode and the entities to be loaded
at boot.

This property sets the field of view angle of the camera. The bigger the
angle the wider the camera view.

\begin{lstlisting}[language=Java]
scene.camera.fov=60
\end{lstlisting}

This property is a multiplier for the velocity of the camera in \texttt{FOCUS} mode.

\begin{lstlisting}[language=Java]
scene.camera.focus.vel=1
\end{lstlisting}

This is a list of entities to be loaded at startup. If you remove any of these
you won't have the option to toggle its visibility from the application GUI.

\begin{lstlisting}[language=Java]
scene.entities=earth gaia moon mw
\end{lstlisting}

\section{Program options}
Finally, you can disable the tutorial windows being shown every time the
application is started with the following property.

\begin{lstlisting}[language=Java]
program.tutorial=true
\end{lstlisting}

\chapter{Running and operating instructions}

\section{Running the program}
In order to run the program follow the instructions of your operating
system.

\subsection{Linux}
In order to run the application on Linux, open the terminal, give execution
permissions to the run.sh file and then run it.

\begin{lstlisting}[language=bash,basicstyle={\tiny\ttfamily}]
> cd path_to_gaiasandbox_folder/
> chmod +x run.sh
> run.sh
\end{lstlisting}

Alternatively you can run the jar file directly, specifying the configuration
file.

\begin{lstlisting}[language=bash,basicstyle={\tiny\ttfamily}]
> java -Dproperties.file=conf/global.properties -jar gaiasandbox.jar
\end{lstlisting}

\subsection{Windows}
In order to run the application on Windows, open a terminal window (write
'cmd' in the start menu search box) and run the run.bat file.

\begin{lstlisting}[language=bash,basicstyle={\tiny\ttfamily}]
> cd path_to_gaiasandbox_folder\
> run.bat
\end{lstlisting}
	
Alternatively you can run the jar file directly, specifying the configuration
file.

\begin{lstlisting}[language=bash,basicstyle={\tiny\ttfamily}]
> java -Dproperties.file=conf/global.properties -jar gaiasandbox.jar
\end{lstlisting}

\subsection{MacOS}
To run the application on MacOS systems, run the jar file specifying the
configuration file.

\begin{lstlisting}[language=bash,basicstyle={\tiny\ttfamily}]
> java -Dproperties.file=conf/global.properties -jar gaiasandbox.jar
\end{lstlisting}

\section{Operating instructions}

\subsection{User interface}
The Gaia Sandbox application has an on-screen user interface designed to be
easy to use. It is divided into three sections, Time, Camera and Object
visibility.

\subsubsection{Time}
You can play and pause the simulation using the \texttt{PLAY/PAUSE} button in the
\texttt{OPTIONS} window to the left. You can also change the pace, which is the 
simulation time to real time ratio, expressed in $h/sec$. If the pace is 2.1, then
one second of real time translates to two hours of simulation time.
Finally, the current simulation date is given in the bottom box of
the Time group.

\subsubsection{Camera}
In the camera options pane on the left you can select the type of camera. 
This can also be done by using the Numpad 0-8 keys.
There is also a list of focus objects that can be selected from
the interface. When an object is selected the camera automatically centers
it in the view and you can rotate around it or zoom in and out.
Objects can also be selected by clicking on them directly in the view. \\
** Hint: Try focusing on Gaia and zoom in to inspect its movement and orbit.

\subsubsection{Object visibility}
Most graphical elements can be turned off and on using the visibility
toggles at the bottom of the \texttt{OPTIONS} window. For example you can remove
the stars from the display by clicking on the 'stars' toggle. Some 
graphical elements are a bit costly to display, such as the 'star names'
or the 'Milky Way'. If you experience poor frame rates (below 30 FPS), try
disabling some of the graphical elements and see if this fixes the issue.

\subsection{Controls}
This section describes the controls of the Gaia Sandbox.

\subsubsection{Keyboard controls}

\begin{center}
    \begin{tabular}{ | r | p {11 cm} |}
	\hline
	\textbf{Control} & \textbf{Action} \\ \hline
	Numpad 0-8 & Change camera mode \\
	Numpad 0 & Sets camera mode to \texttt{FREE} \\
	Numpad 1 & Sets camera mode to \texttt{FOLLOW\_ORBIT} \\
	Numpad 2 & Sets camera mode to \texttt{FOLLOW\_GAIA} \\
	Numpad 3 & Sets camera mode to \texttt{GAIA\_INTERTIAL\_RF}, where 
			the camera is fixed with respect to Gaia \\
	Numpad 4 & Sets camera mode to \texttt{ROTATE\_GAIA}, where the camera rotates with Gaia \\
	Numpad 5 & Sets camera mode to \texttt{ROTATE\_PRECESS\_GAIA}, where the camera rotates and precesses with Gaia \\
	Numpad 6 & Sets camera mode to \texttt{INSIDE\_GAIA} \\
	Numpad 7 & Sets camera mode to \texttt{SCENE}, where the camera tries to get Gaia and the Earth in its field of view \\
	Numpad 8 & Sets camera mode to \texttt{FOCUS}, where the camera focuses the selected object \\
	P & Toggle simulation play/pause \\
	I & Toggle reference axes visibility \\
	\hline
    \end{tabular}
\end{center}

\subsubsection{Mouse controls}

\begin{center}
    \begin{tabular}{ | r | p {9 cm} |}
	\hline
	\textbf{Control} & \textbf{Action} \\ \hline
	Left click on object & Select object as focus \\
	Left click + drag & Pitch and yaw (\texttt{FREE} mode) or rotate around foucs (\texttt{FOCUS} mode) \\
	Middle click + drag or wheel & Forward/backward movement \\
	Right click + drag & Move sideways (only in \texttt{FREE} mode) \\
	Shift + left click + drag & Camera roll \\
	\hline
    \end{tabular}
\end{center}

\chapter{Copyright and licensing information}
This software is published and distributed under the \texttt{LGPL}
(Lesser General Public License) license. You can find the full license
text in the \texttt{LICENSE.txt} file or visiting 
\href{https://www.gnu.org/licenses/lgpl-3.0-standalone.html}{www.gnu.org/licenses/lgpl-3.0-standalone.html}.

\chapter{Contact information}
The main webpage of the project is \href{http://www.zah.uni-heidelberg.de/gaia2/outreach/gaiasandbox}{www.zah.uni-heidelberg.de/gaia2/outreach/gaiasandbox}.
There you can find the latest versions and the latest information on the Gaia Sandbox.
\section{Main designer and developer}
Toni Sagrist\`a Sell\'es
\begin{itemize}
\item E-mail: \href{mailto:tsagrista@ari.uni-heidelberg.de}{tsagrista@ari.uni-heidelberg.de}
\item Personal webpage: \href{http://www.tonisagrista.com}{www.tonisagrista.com}
\end{itemize}

\section{Contributors}
Dr. Stefan Jordan
\begin{itemize}
\item E-mail: \href{mailto:jordan@ari.uni-heidelberg.de}{jordan@ari.uni-heidelberg.de}
\item Personal webpage: \href{http://www.stefan-jordan.de}{www.stefan-jordan.de}
\end{itemize}


\chapter{Known bugs, issues and TODOs}
\begin{itemize}
\item Distances are not unified amongst all the entities.
\item The program crashes if Gaia reaches the end or the beginning
of the orbit line.
\item Clipping problems when rendering to files in models with very near 
surfaces (i.e. bottom of Gaia). Possible depth buffer problem.
\item If the camera is far from the orbit line and FOLLOW\_ORBIT or 
FOLLOW\_GAIA camera modes are selected, the camera should quickly
move to the orbit position.
\item Possible problems with the Mac and Windows versions.
\item OwnTextButton cursor image (pointing hand) disappears when button is
pressed.
\item Accented and greek letters must be added to the character set.
\item Some missing symbols (such as the degree symbol) must be added
to the character set.
\end{itemize}

\chapter{Credits and acknowledgements}
The author would like to acknowledge the following people, or the
people behind the following technologies/resources:

\begin{itemize}
\item The DLR (\href{http://www.dlr.de/}{http://www.dlr.de/}) for financing this project.
\item Dr. Martin Altmann for providing the orbit data.
\item Libgdx - \href{http://libgdx.badlogicgames.com}{http://libgdx.badlogicgames.com}
\item HYG catalog - \href{http://www.astronexus.com/hyg}{http://www.astronexus.com/hyg}
\end{itemize}

\chapter{Change log}

\textbf{[version 0.504b] - 16/04/2014}
\begin{itemize}
\item Fixed Sun distance in FOCUS mode.
\item Log messages hidden by default.
\item Smooth shading for all models except mw.
\item Normals removed from mw model. Transparency tweaked.
\item Initial camera position moved to Gaia.
\end{itemize}

\textbf{[version 0.503b] - 15/04/2014}
\begin{itemize}
\item Added distance to focus.
\item Fixed Java8 library issue.
\item Fixed font shader for MacOS.
\end{itemize}

\textbf{[version 0.502b] - 14/04/2014}
\begin{itemize}
\item First public version.
\end{itemize}


\end{document}
